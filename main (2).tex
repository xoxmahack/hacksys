 \documentclass{article}
\usepackage{hyperref}
\usepackage{graphicx} % Required for inserting images
\usepackage[russian]{babel}
\title{Конспект по теме "теорема Пифагора"}
\author{Автор: Великий Хохма}
\date{19 сентября 2025}

\begin{document}

\maketitle


\tableofcontents
\newpage
\section{Введение}
Теорема Пифагора — одна из важнейших теорем евклидовой геометрии. Она находит
применение в самых разных областях:
\begin{itemize}
    \item геометрия и тригонометрия
    \item физика
    \item инженерные расчёты
    \item компьютерная графика
\end{itemize}

\section{Формулировка теоремы}
\textbf{Слова:}
В прямоугольном треугольнике квадрат гипотенузы равен сумме квадратов
катетов.
\begin{equation}\label{eq:pythago}
c^2 = a^2 + b^2
\end{equation}
\begin{center}

  \text Как видно из формулы \hyperlink{important}{\textbf{1}}, знание двух сторон позволяет найти третью.
\end{center}

\section{Доказательство(набросок)}
\begin{center}
    \text {Одно из доказательств основывается на площади квадрата, составленного
из четырёх одинаковых прямоугольных треугольников и малого квадрата
в центре. Раскладывая площадь двумя способами, получаем c2 = a2 + b2.}
\end{center}

\section{Примеры расчёта}
\hypertarget{important}{\textbf{Пример 1}}

\begin{center}

  \text a = 3, b = 4


  \text $ c = \sqrt{a^2 + b^2} = \sqrt{9 + 16} = 5$
\end{center}
\textbf{Пример 2}

1. Дано: a = 5, b = 12

2. Решение:
\begin{center}

  \text $ c = \sqrt{5^2 + 12^2} = \sqrt{25 + 144} = 5$
\end{center}
\section{Таблица значений}
\begin{center}
\begin{tabular}{|c|c|c|}
\hline
Катет а & Катет б & Гипотенуза с \\
\hline
3 & 4 & 5 \\
\hline
5 & 12 & 13 \\
\hline
7 & 24 & 25 \\
\hline
\end{tabular}
\end{center}

\section{Иллюстрация}
Ниже пример изображения (не забудьте добавить файл triangle.png в ту же папку):

\begin{center}
\includegraphics[width=0.5\linewidth]{triangle.png}
\end{center}


\section{Заключение}

Теорема Пифагора — один из краеугольных камней геометрии, помогающий решать
множество практических задач.


\section{Ссылки и литература}
\begin{itemize}
    \item Википедия: \href{https://ru.wikipedia.org/wiki/%D0%A2%D0%B5%D0%BE%D1%80%D0%B5%D0%BC%D0%B0_%D0%9F%D0%B8%D1%84%D0%B0%D0%B3%D0%BE%D1%80%D0%B0}{ Теорема Пифагора}
    \item Классические учебники геометрии

\end{itemize}
\end{document}
